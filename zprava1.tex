\documentclass[12pt]{article}
%\usepackage{epsf,epic,eepic,eepicemu}
%\documentstyle[epsf,epic,eepic,eepicemu]{article}
%\usepackage[cp1250]{inputenc}
\usepackage[utf8]{inputenc}
\usepackage[czech, english]{babel}
\usepackage{czech}
\usepackage[T1]{fontenc} 
\usepackage{verbatim}
\usepackage{graphicx}
\usepackage{hyperref}
\usepackage{lmodern}
\usepackage{float}



\begin{document}
%\oddsidemargin=-5mm \evensidemargin=-5mm \marginparwidth=.08in
%\marginparsep=.01in \marginparpush=5pt \topmargin=-15mm
%\headheight=12pt \headsep=25pt \footheight=12pt \footskip=30pt
%\textheight=25cm \textwidth=17cm \columnsep=2mm \columnseprule=1pt
%\parindent=15pt\parskip=2pt

\begin{center}
\bf Semestrální projekt MI-PAP 2010/2011:\\[5mm]
    Paralelní řadící algoritmy\\[5mm]
    Pavel Benáček\\   
    Tomáš Čejka\\[2mm]
magisterské studijum, FIT ČVUT, Kolejní 550/2, 160 00 Praha 6\\[2mm]
\today
\end{center}

\section{Definice problému a popis sekvenčního algoritmu}
Podle zadání semestrální práce je úkolem implementace aspoň tří z následujících
řadících algoritmů: Shearsort, 3D sort, Sudo-lichý Mergesort a Bitonic sort.

Pro řešení jsme si vybrali algoritmy Shearsort, Sudo-lichý Mergesort a Bitonic sort.

Pro účely měření času běhu výsledných programů a porovnání paralelních algoritmů se sekvenčním
řazením jsme si implementovali sekvenční algoritmus Mergesort a ten vzali jako referenční sekvenční
řešení.

Semestrální práce spočívala v implementaci vybraných řadících algoritmů jako sekvenční řešení, paralelní
řešení na počítači se sdílenou pamětí s použítím knihovny OpenMP a následně s využitím technologie CUDA
společnosti NVIDIA$^{\tiny{\textregistered}}$.

\section{Popis paralelního algoritmu a jeho implementace v OpenMP}
\subsection{Sudo-lichý mergesort}
Algoritmus Sudo-lichý mergesort je navržen na strukturu řadící sítě, kterou si můžeme představit jako
systém s distribuovanou pamětí, který obsahuje množinu procesorů uspořádaných do mřížky a vhodně pospojovaných.

Algoritmus funguje tak, že na vstup systému se přivede množina dat, která se během průchodu sítí seřadí.

Řešení paralelizace s knihovou OpenMP pracuje na počítači se sdílenou pamětí, není potřeba množinu dat
rozdělovat na dílčí části a distribuovat/rozkopírovat je mezi procesory. Místo toho jsou veškerá data
umístěna v paměti v jednom poli a jednotlivá paralelní vlákna přistupují k disjunktním oblastem sdíleného
pole.

Snažili jsme se o co nejrovnoměrnější rozdělení dat mezi vlákna, takže data virtuálně rozdělíme na začátku
programu na stejné části, jejichž počet je rovný nějaké mocnině dvou a jednotlivá vlákna seřadí tuto podposloupnost
dat. Tento přístup je možný díky rekurzivní povaze algoritmu a řadících sítí.
Po seřazení podposloupností je potřeba ještě provést jejich sloučení. Sloučení se provádí v logaritmickém čase,
protože .....
\subsection{Bitonic sort}

\subsection{Shearsort}

\section{Naměřené výsledky a vyhodnocení}
\subsection{Způsob měření}

\subsection{Naměřené časy}

\begin{table}[H]
\begin{center}
\begin{tabular}{|r|r|r|r|}
\hline
Počet CPU & A & B & C\\
\hline
1 & 321,8 & 307,3 & 338,6\\
\hline
2 & 38,95370 & 35,46150 & 40,27100\\
\hline
4 & 12,20080 & 12,45520 & 21,30230\\
\hline
8 & 7,14178 & 10,43290 & 15,05280\\
\hline
16 & 6,55517 & 8,99971 & 11,43290\\
\hline
24 & 6,21340 & 7,14230 & 8,85400\\
\hline
32 & 6,24150 & 7,13790 & 8,42100\\
\hline
\end{tabular} 
\end{center}
\caption{Propojovací síť InfiniBand}
\end{table}

\subsection{Spočítané zrychlení}
\begin{table}[H]
\begin{center}
\begin{tabular}{|r|r|r|r|}
\hline
Počet CPU & A & B & C\\
\hline
1 & 1 & 1 & 1\\
\hline
2 & 8,15 & 8,78 & 8,99\\
\hline
4 & 24,40 & 24,84 & 16,55\\
\hline
8 & 41,18 & 29,26 & 23,92\\
\hline
16 & 42,81 & 35,45 & 31,66\\
\hline
24 & 45,03 & 42,86 & 37,77\\
\hline
32 & 46,48 & 42,32 & 38,55\\
\hline
\end{tabular}
\end{center}
\caption{Zrychlení Propojovací síť Ethernet}
\end{table}

\begin{table}[H]
\begin{center}
\begin{tabular}{|r|r|r|r|}
\hline
Počet CPU & A & B & C\\
\hline
1 & 1 & 1 & 1\\
\hline
2 & 8,26 & 9,00 & 8,99\\
\hline
4 & 26,38 & 25,61 & 16,99\\
\hline
8 & 45,06 & 30,58 & 24,05\\
\hline
16 & 49,09 & 35,45 & 31,66\\
\hline
24 & 51,79 & 44,66 & 40,89\\
\hline
32 & 51,56 & 44,69 & 42,99\\
\hline
\end{tabular} 
\end{center}
\caption{Zrychlení Propojovací síť InfiniBand}
\end{table}
Zrychlení nám vyšlo superlineární pro všechna měření paralelní úlohy, 
protože platí $$S > p$$, kde \(S\) je zrychlení a \(p\) je počet procesorů. Důvodem takového zrychlení je vhodné rozesílání
nejlepších dosažených výsledků a tím i četné ořezávání stavového prostoru.

\subsection{Vyhodnocení}

\section{Závěr}

\section{Literatura}
\begin{enumerate}
\item \href{https://edux.fit.cvut.cz/courses/MI-PAR/labs/prohledavani_do_hloubky}{Stránky cvičení - prohledávání do hloubky}
\item \href{https://edux.fit.cvut.cz/courses/MI-PAR/labs/poznamky_k_implementaci}{Stránky cvičení - poznámky k implementaci}
\end{enumerate}


\appendix
\section{Grafy závislosti času na počtu vláken}
\subsection{Propojovací síť InfiniBand}
Grafy závislosti času na počtu procesorů při použití propojovací sítě InfiniBand.
\begin{figure}[H]
\begin{center}
%\includegraphics[width=8cm]{grafy-zprava/testAinfib.png}
\caption{Test A}
\label{fig:testAinfib}
\end{center}
\end{figure}

\begin{figure}[H]
\begin{center}
%\includegraphics[width=8cm]{grafy-zprava/testBinfib.png}
\caption{Test B}
\label{fig:testBinfib}
\end{center}
\end{figure}

\begin{figure}[H]
\begin{center}
%\includegraphics[width=8cm]{grafy-zprava/testCinfib.png}
\caption{Test C}
\label{fig:testCinfib}
\end{center}
\end{figure}

\subsection{Propojovací síť Ethernet}
Grafy závislosti času na počtu procesorů při použití propojovací sítě Ethernet.
\begin{figure}[H]
\begin{center}
%\includegraphics[width=8cm]{grafy-zprava/testAeth.png}
\caption{Test A}
\label{fig:testAether}
\end{center}
\end{figure}

\begin{figure}[H]
\begin{center}
%\includegraphics[width=8cm]{grafy-zprava/testBeth.png}
\caption{Test B}
\label{fig:testBether}
\end{center}
\end{figure}

\begin{figure}[H]
\begin{center}
%\includegraphics[width=8cm]{grafy-zprava/testCeth.png}
\caption{Test C}
\label{fig:testCether}
\end{center}
\end{figure}

\section{Grafy zrychlení}
Grafy zobrazující závislost zrychlení výpočtu na počtu procesorů.
\begin{figure}[H]
\begin{center}
%\includegraphics[width=8cm]{grafy-zprava/zrychleniinf.png}
\label{fig:zrychleniinf}
\caption{InfiniBand}
\end{center}
\end{figure}

\begin{figure}[H]
\begin{center}
%\includegraphics[width=8cm]{grafy-zprava/zrychlenieth.png}
\label{fig:zrychlenieth}
\caption{Ethernet}
\end{center}
\end{figure}

\end{document}
